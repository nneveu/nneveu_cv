%% start of file `template.tex'.
%% Copyright 2006-2013 Xavier Danaux (xdanaux@gmail.com).
%
% This work may be distributed and/or modified under the
% conditions of the LaTeX Project Public License version 1.3c,
% available at http://www.latex-project.org/lppl/.


\documentclass[11pt,a4paper,sans]{moderncv}        % possible options include font size ('10pt', '11pt' and '12pt'), paper size ('a4paper', 'letterpaper', 'a5paper', 'legalpaper', 'executivepaper' and 'landscape') and font family ('sans' and 'roman')

% moderncv themes
\moderncvstyle{banking}         % style options are 'casual' (default), 'classic', 'oldstyle' and 'banking'
\moderncvcolor{blue}          % color options 'blue' (default), 'orange', 'green', 'red', 'purple', 'grey' and 'black'
%\renewcommand{\familydefault}{\sfdefault}         % to set the default font; use '\sfdefault' for the default sans serif font, '\rmdefault' for the default roman one, or any tex font name
%\nopagenumbers{}                                  % uncomment to suppress automatic page numbering for CVs longer than one page

% character encoding
\usepackage[utf8]{inputenc} % if you are not using xelatex ou lualatex, replace by the encoding you are using
%\usepackage{hyperref}
\newcommand{\lsedit}[1]{\textsf{{\color{violet}{ LS note:}   #1 }}}
\newcommand{\lsnote}[1]{\textsf{{\color{violet}{ LS note:}   #1 }}}
\newcommand{\nrnote}[1]{\textsf{{\color{violet}{ NN note:}   #1 }}}
\usepackage{multicol}
% adjust the page margins
\usepackage[margin=0.7in]{geometry}
%\setlength{\hintscolumnwidth}{3cm}                % if you want to change the width of the column with the dates
%\setlength{\makecvtitlenamewidth}{10cm}           % for the 'classic' style, if you want to force the width allocated to your name and avoid line breaks. be careful though, the length is normally calculated to avoid any overlap with your personal info; use this at your own typographical risks...
                           
%\address{9700 Cass Avenue}{Lemont, IL 60439}{USA}
%\phone[fixed]{+1~(630)~252~7333}                               
%\email{nneveu@hawk.iit.edu}
%\homepage{www.johndoe.com}                         
%\quote{Some quote}                                 

% to show numerical labels in the bibliography (default is to show no labels); only useful if you make citations in your resume
%\makeatletter
%\renewcommand*{\bibliographyitemlabel}{\@biblabel{\arabic{enumiv}}}
%\makeatother
%\renewcommand*{\bibliographyitemlabel}{[\arabic{enumiv}]}% CONSIDER REPLACING THE ABOVE BY THIS

% bibliography with mutiple entries
\usepackage[resetlabels,labeled]{multibib}
\newcites{inprog}{Publications in Progress}
\newcites{refproc}{Refereed Proceedings}
\newcites{proc}{Proceedings}
\newcites{other}{Proceedings}

%--------------------------------------------------------------------------------
\name{Nicole}{Neveu}
\extrainfo{PhD Candidate in Physics, IIT \& ANL}

\begin{document}

\iffalse
\recipient{Accelerator Technology and Applied Physics Division (ATAP)}{Berkeley Lab (LBNL) \\ 1 Cyclotron Road \\ Berkeley, CA 94720}
\date{May, 2018}
\opening{To Whom It May Concern,}
\closing{Thanks for your consideration,\vspace{3em}}
%\enclosure[Attached]{curriculum vit\ae{}}          
\makelettertitle
%\vspace{-4em}

As a PhD candidate in physics at the Illinois Institute of Technology (IIT), 
I am carrying out my thesis work
at the Argonne Wakefield Accelerator (AWA) facility. 
I expect to defend by September 2018, 
with my thesis centered on beam line design, simulation, optimization, 
and experimental preparation of staged two beam acceleration (TBA) at the AWA. 
During this work, 
I have enjoyed learning about computational accelerator physics. 
I am interested in the opening to model collective effects at ATAP, because it could give me the opportunity to continue in this field. 

I have modeled both high and low charge photoinjectors at AWA as well as 
a thermionic gun developed by Euclid Techlabs. My recent simulation 
work takes place on Bebop, 
a HPC cluster provided by the Laboratory Computing Resource Center (LCRC) 
at Argonne National Laboratory (ANL). 
My primary focus is on beam line design 
and optimization for high charge (40~nC) TBA experiments.
I also preform experimental measurements to verify and improve 
our model and control of high charge beams.

I use the parallel, open source, and 3D particle-in-cell code OPAL-T.  
I have been collaborating with the developers by contributing to their 
python tools and aiding in testing of new features.
I also use the open source parallel python code developed by Jeff Larson (ANL) and others in the library libensemble.  
I use this library, in combination with generating functions I write 
to manage multiple parallel instances of OPAL-T running in the same batch job to perform large scale ensemble optimizations or sampling.  
I have used the built in genetic algorithm and parallel framework that is 
incorporated into OPAL and the python nlopt package for optimizations. 
I have also begun investigating surrogate models for accelerators based on supervised learning and polynomial chaos in collaboration with Andreas Adelmann (PSI) and Auralee Edelen (SLAC). 
The goal of this work is to aid  accelerator operations by
decreasing the time needed to manually tune the machine.  


My colleagues and I have installed OPAL-T on the ANL HPC machines Bebop and Blues. More recently, we were awarded a Director's Discretionary allocation to extend our work on Theta, an Argonne Leadership Computing Facility (ALCF) machine. Our work on Theta focuses on high fidelity simulations,
including scaling studies and preparations for realistic, one-to-one simulations
for hand off to QuickPIC simulations in collaboration with the UCLA plasma wakefield group.

My past research experience aligns well with the ongoing work at ATAP.  My aim is to continue learning about the field of computational accelerator physics, and contribute to this field in the future. As my career unfolds, I also hope to continue contributing to open source projects in a sustainable and forward thinking way.


\vspace{3em}

\makeletterclosing

\clearpage
\fi

\name{Nicole}{Neveu}
\title{CV}                               % optional, remove / comment the line if not wanted
\address{9700 Cass Avenue}{Lemont, IL 60439}{USA}% optional, remove / comment the line if not wanted; the "postcode city" and and "country" arguments can be omitted or provided empty
%\phone[mobile]{+1~(234)~567~890}                  
\phone[fixed]{+1~(630)~252~7333}                   
%\phone[fax]{+3~(456)~789~012}                                                  
\email{nneveu@hawk.iit.edu}
\extrainfo{nneveu@anl.gov}
\makecvtitle

\section{Education}
\cventry{Chicago, IL}{Illinois Institute of Technology}{Ph.D Candidate in Physics}{2013--Present}{College of Science}{}%{\textit{Grade}}{Description}  % arguments 3 to 6 can be left empty
\cventry{Houston, TX}{University of Houston}{B.S. Electrical Engineering}{2009--2013}{Cullen College of Engineering}{}

\iffalse
\section{Ph.D Thesis}
\cvitem{Title}{\emph{Towards Staged Two Beam Acceleration at the Argonne Wakefield Accelerator}}
\cvitem{Advisors}{Linda Spentzouris, John Power}
\cvitem{Summary}{
	I discuss beam line design, simulation, and optimization, of a 
	beam line with the potential for dielectric two beam acceleration.
    Preliminary and prepratory experimental measurements
    are included.    }
\fi

\section{Research Experience}
\cventry{Lemont, IL}{Argonne National Laboratory (ANL)}{Graduate Student, Argonne Wakefield Accelerator (AWA)}{2013--Present}{}{
Optimization, and simulation of rf photoinjectors at AWA using 
the parallel PIC code, OPAL-T, \newline and experimental validation of results. Completed and ongoing work includes:
\begin{itemize}%
	\item Thesis: Design for Staged Two Beam Acceleration (TBA) at the Argonne Wakefield Accelerator
	\begin{itemize}
		\item Simulations and experimental preparations for implementation of staged TBA at AWA.
	\end{itemize}
	\item Simulations:
	\begin{itemize}%
		\item TBA beam line design at 40 nC for AWA (in progress).
		\item Collaborating with Auralee Edelen (SLAC) and Andreas Adelmann (PSI) on using surrogate models for accelerators.
		\item Collaborating with Jeff Larson (ANL) on using novel optimization algorithms for accelerators. 
		\item Large ensemble HPC simulations (for work listed above).
		\item High charge linac optimization (40 nC) using nlopt python package.
		\item Optical Transistion Radiation (OTR) simulations at 1 nC.
		\item Simulation of the Euclid Techlabs thermionic gun design.
		\item Wrote post processing script in python for analyzing transverse beam images.
		\item Built and installed OPAL-T on two Linux clusters at ANL: Bebop and Theta.
		\item \url{https://github.com/nneveu}
	\end{itemize}
	\item Experimental:
	\begin{itemize}%		
		\item Beam size, charge, and energy measurements.
		\item RF power measurements for linac tanks and gun.
		\item 40 nC emittance and bunch length measurements.
		\item Intensity improvement of UV laser pulse train. 
		\item UV multisplitter assembly and characterization.
		\item Design of relay imaging transport system for drive beam line.
	\end{itemize}
\end{itemize}
}
\section{Publications}
% Publications from a BibTeX file without multibib
%  for numerical labels: \renewcommand{\bibliographyitemlabel}{\@biblabel{\arabic{enumiv}}}% CONSIDER MERGING WITH PREAMBLE PART
%  to redefine the heading string ("Publications"): \renewcommand{\refname}{Articles}
%\nocite{*}
%\bibliographystyle{plain}
%\bibliography{publications}                        % 'publications' is the name of a BibTeX file

\nociteinprog{*}
\bibliographystyleinprog{ieeetr}
\bibliographyinprog{./cvparts/inprog}

\nociterefproc{*}
\bibliographystylerefproc{ieeetr}
\bibliographyrefproc{./cvparts/refproc}

%\nociteproc{*}
%\bibliographystyleproc{ieeetr}
%\bibliographyproc{./cvparts/proceedings}

\section{Research Presentations}

\subsection{Invited Conference or Workshop} %Workshops, Symposium
\textbf{Study of space charge dominated beams at the AWA rf photoinjector}\newline
Space Charge 2017, Darmstadt, Germany. October 6, 2017

\subsection{Seminar and Colloquia}
%\subsection{Other Talks}
\textbf{Photoinjector simulations using OPAL-T for the Argonne Wakefield Accelerator}\newline
Beam Dynamics Palaver AMAS, Paul Scherrer Institute, Villigen, Switzerland. May 23, 2017 
\vspace{0.3em}

\textbf{Research at the Argonne Wakefield Accelerator}\newline
Young Scientist Symposium, HEP ANL, Lemont, IL. May 17, 2016 
\vspace{0.3em}

\textbf{Two Beam Acceleration at the Argonne Wakefield Accelerator}\newline
Center for Accelerator and Particle Physics, IIT (CAPP), Chicago, IL. April 28, 2016 

\section{Awards}
\cventry{}{Selected to attend this summer.}{Argonne Training Program on Extreme Scale Computing (ATPESC)}{August 2018}{}{}

\cventry{}{Two million core hours awarded on Theta, for work done in collaboration with PSI.}{ANL Dircetor's Discretionary Allocation}{ 2018}{}{}%}
\cventry{}{Funds granted to travel and present work at IPAC 2018 in Canada.}{IPAC 2018 Travel Award}{April 2018}{}{}%}
\cventry{}{Funds granted to travel and present work at IPAC 2017 in Denmark.}{IPAC 2017 Travel Award}{May 2017}{}{}%}
\cventry{}{Funds granted to attend a short course on beam dynamics at NAPAC 2016.}{Phelps Grant Award}{Nov 2016}{}{}%}
\cventry{}{One year grant funded by DOE; for TBA work done at ANL.}{Science Graduate Student Research Award}{2015--2016}{}{}%}

\cventry{}{Attended the following clases:}{USPAS Scolarships}{2014-2018}{}{}
%\section{USPAS Courses Taken}
\begin{itemize}
	\item {Microwave Measurements and Beam Instrumentation}
	\item {Fundamentals of Timing and Synchronization}
	\item {Vacuum Science and Technology}
	\item {Accelerator Physics Using Maple}
\end{itemize}


\section{Programming Languages}
\cvlistdoubleitem{\cvitem{Proficient in}{OPAL-T, Python, Matlab}}{\cvitem{Prior experience with}{C/C++, Fortran, Perl, SQL}}


\section{Teaching}
\cventry{East Lansing, MI}{TA, Fundamentals of Accelerator Physics and Technology}{USPAS hosted by Michigan State University}{June 2018}{}{}

\cventry{Chicago, IL}{Adjunct, Physics of Sound and Light}{Vandercook School of Music}{Spring 2018}{}{}%}

\cventry{Lanzhou, China}{TA, Technical English}{Institute of Modern Physics}{July 2017}{}{}%}

\cventry{Beijing, China}{TA, Technical English}{Tsinghua University}{June 2015, 2016, 2017}{}{}%}

\cventry{New Brunswick, NJ}{TA, Fundamentals of Accelerator Physics and Technology}{USPAS hosted by Rutgers University}{June 2015}{}{}

\cventry{Chicago, IL}{TA, Electromagnetism and Optics \& Instrumentation Laboratory}{Illinois Institute of Technology}{2013--2015}{}{}

\iffalse
\section{Recent Volunteer work}
\cventry{}{Hosted by ANL, participated as a mentor}{Introduce a girl to engineering day}{April 2017, Feb 2018}{}{}
\cventry{}{Visited Spry CLHS and Glenbard East High School.}{Hour of Code}{Dec 2017}{}{}
\cventry{}{Answered questions from high school students online.}{Adopt-a-Physicist: \url{https://www.adoptaphysicist.org/webdocs/about.cfm}{}}{Oct 2017}{}{}
%\cventry{}{Monroe School}{Science Fair Judge}{Feb 2017}{}{}
\fi

\section{Memberships}
\cvlistdoubleitem{American Physical Society}{Society of Women Engineers}
\cvlistdoubleitem{Institute of Electrical and Electronics Engineers}{}


\iffalse
\section{References}
\vspace{1em}

\begin{enumerate}
	\itemsep1em 
	\item Linda Spentzouris, IIT
	\begin{itemize}
		\item[] Phone: 312.567.3577
		\item[] Email: linda@agni.phys.iit.edu
	\end{itemize}
	\item Andreas Adelmann, PSI
	\begin{itemize}
		\item[] Phone: 41.56.310.4233
		\item[] Email: andreas.adelmann@psi.ch
	\end{itemize}
	\item Jeffrey Larson, MCS-ANL
	\begin{itemize}
		\item[] Phone: 630.252.3221
		\item[] Email: jmlarson@anl.gov
	\end{itemize}
	\item Chunguang Jing, Euclid Techlabs, LLC
	\begin{itemize}
		\item[] Phone: 630.313.6562
		\item[] Email: jingchunguang@gmail.com
	\end{itemize}
	
\end{enumerate}

\fi


\clearpage

\end{document}


%% end of file `template.tex'.
