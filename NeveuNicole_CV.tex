%% start of file `template.tex'.
%% Copyright 2006-2013 Xavier Danaux (xdanaux@gmail.com).
%
% This work may be distributed and/or modified under the
% conditions of the LaTeX Project Public License version 1.3c,
% available at http://www.latex-project.org/lppl/.


\documentclass[11pt,a4paper,sans]{moderncv}        % possible options include font size ('10pt', '11pt' and '12pt'), paper size ('a4paper', 'letterpaper', 'a5paper', 'legalpaper', 'executivepaper' and 'landscape') and font family ('sans' and 'roman')

% moderncv themes
\moderncvstyle{banking}                            % style options are 'casual' (default), 'classic', 'oldstyle' and 'banking'
\moderncvcolor{blue}                                % color options 'blue' (default), 'orange', 'green', 'red', 'purple', 'grey' and 'black'
%\renewcommand{\familydefault}{\sfdefault}         % to set the default font; use '\sfdefault' for the default sans serif font, '\rmdefault' for the default roman one, or any tex font name
%\nopagenumbers{}                                  % uncomment to suppress automatic page numbering for CVs longer than one page

% character encoding
\usepackage[utf8]{inputenc}                       % if you are not using xelatex ou lualatex, replace by the encoding you are using
%\usepackage{CJKutf8}                              % if you need to use CJK to typeset your resume in Chinese, Japanese or Korean

% adjust the page margins
\usepackage[margin=0.7in]{geometry}
%\setlength{\hintscolumnwidth}{3cm}                % if you want to change the width of the column with the dates
%\setlength{\makecvtitlenamewidth}{10cm}           % for the 'classic' style, if you want to force the width allocated to your name and avoid line breaks. be careful though, the length is normally calculated to avoid any overlap with your personal info; use this at your own typographical risks...

% personal data
\name{Nicole}{Neveu}
\title{CV}                               % optional, remove / comment the line if not wanted
\address{9700 Cass Avenue}{Lemont, IL 60439}{USA}% optional, remove / comment the line if not wanted; the "postcode city" and and "country" arguments can be omitted or provided empty
%\phone[mobile]{+1~(234)~567~890}                   % optional, remove / comment the line if not wanted
\phone[fixed]{+1~(630)~252~7333}                    % optional, remove / comment the line if not wanted
%\phone[fax]{+3~(456)~789~012}                      % optional, remove / comment the line if not wanted
\email{nneveu@anl.gov}                               % optional, remove / comment the line if not wanted
\email{nneveu@hawk.iit.edu}
%\homepage{www.johndoe.com}                         % optional, remove / comment the line if not wanted
%\extrainfo{additional information}                 % optional, remove / comment the line if not wanted
%\photo[64pt][0.4pt]{picture}                       % optional, remove / comment the line if not wanted; '64pt' is the height the picture must be resized to, 0.4pt is the thickness of the frame around it (put it to 0pt for no frame) and 'picture' is the name of the picture file
%\quote{Some quote}                                 % optional, remove / comment the line if not wanted

% to show numerical labels in the bibliography (default is to show no labels); only useful if you make citations in your resume
%\makeatletter
%\renewcommand*{\bibliographyitemlabel}{\@biblabel{\arabic{enumiv}}}
%\makeatother
%\renewcommand*{\bibliographyitemlabel}{[\arabic{enumiv}]}% CONSIDER REPLACING THE ABOVE BY THIS

% bibliography with mutiple entries
\usepackage[resetlabels,labeled]{multibib}
\newcites{proc}{Proceedings}
\newcites{refproc}{Refereed Proceedings}
\newcites{inprog}{Publications in Progress}
%--------------------------------------------------------------------------------

\begin{document}
%\begin{CJK*}{UTF8}{gbsn}                          % to typeset your resume in Chinese using CJK
%-----       resume       ---------------------------------------------------------
\makecvtitle

\section{Education}
\cventry{Chicago, IL}{Illinois Institute of Technology}{Ph.D Candidate in Physics}{2013--Present}{College of Science}{}%{\textit{Grade}}{Description}  % arguments 3 to 6 can be left empty
\cventry{Houston, TX}{University of Houston}{B.S. Electrical Engineering}{2009--2013}{Cullen College of Engineering}{}

\section{Ph.D Thesis}
\cvitem{Title}{\emph{Design for Staged Two Beam Acceleration at the Argonne Wakefield Accelerator}}
\cvitem{Advisors}{Linda Spentzouris, John Power}
\cvitem{Summary}{
	Staged two beam acceleration using dielectric structures has yet to 
	be achieved anywhere in the world. In this thesis, I discuss beam 
	line design, simulation, and optimization, of a 
	beam line with the potential for dielectric two beam acceleration.
    Preliminary and prepratory experimental measurements
    are included.    }


\section{Research Experience}
\cventry{Lemont, IL}{Argonne National Laboratory (ANL)}{Graduate Student, Argonne Wakefield Accelerator (AWA)}{2013--Present}{}{
Design, optimization, and simulation of rf photoinjectors at AWA using OPAL-t.\newline
Experimental validation of results. Completed and ongoing work includes:
\begin{itemize}%
	\item Simulations and Software:
	\begin{itemize}%
		\item Two beam acceleration (TBA) beam line design  
		\item Installed OPAL-T on two Linux clusters at ANL: Bebop and Theta
		\item Wrote beam image processing script in python.
		\item High charge linac optimization (40 nC).
		\item Collaborating with Jeff Larson (ANL) on using novel optimization algorithms.
		\item Wrote optimization scripts in python using nlopt package.
		\item Optical Transistion Radiation (OTR) at 1nC
		\item Thermionic RF gun, designed by Euclid TechLabs
	\end{itemize}
	\item Experimental and hardware:
	\begin{itemize}%		
		\item Beam size, charge, and energy measurements
		\item RF power measurements for linac tanks and gun.
		\item 40nC linac emittance and bunch length measurements.
		\item Intensity improvement of UV laser pulse train 
		\item UV multisplitter assembly and characterization
		\item Design of relay imaging transport system for drive beam line.
	\end{itemize}
\end{itemize}
}
\section{Publications}
% Publications from a BibTeX file without multibib
%  for numerical labels: \renewcommand{\bibliographyitemlabel}{\@biblabel{\arabic{enumiv}}}% CONSIDER MERGING WITH PREAMBLE PART
%  to redefine the heading string ("Publications"): \renewcommand{\refname}{Articles}
%\nocite{*}
%\bibliographystyle{plain}
%\bibliography{publications}                        % 'publications' is the name of a BibTeX file

\nociterefproc{*}
\bibliographystylerefproc{ieeetr}
\bibliographyrefproc{./cvparts/refproc}
%doi :10.1088/1742-6596/874/1/012062
%\subsection{Submitted Proceedings}
\nociteproc{*}
\bibliographystyleproc{ieeetr}
\bibliographyproc{./cvparts/proceedings}

\section{Research Presentations}

\subsection{Invited Conference or Workshop} %Workshops, Symposium
\textbf{Study of space charge dominated beams at the AWA rf photoinjector}\newline
Space Charge 2017, Darmstadt, Germany. October 6, 2017

\subsection{Seminar and Colloquia}
%\subsection{Other Talks}
\textbf{Photoinjector simulations using OPAL-t for the Argonne Wakefield Accelerator}\newline
Beam Dynamics Palaver AMAS, Paul Scherrer Institute, Villigen, Switzerland. May 23, 2017 
\vspace{0.3em}

\textbf{Research at the Argonne Wakefield Accelerator}\newline
Young Scientist Symposium, HEP ANL, Lemont, IL. May 17, 2016 
\vspace{0.3em}

\textbf{Two Beam Acceleration at the Argonne Wakefield Accelerator}\newline
Center for Accelerator and Particle Physics, IIT (CAPP), Chicago, IL. April 28, 2016 


\section{Teaching}
\cventry{East Lansing, MI}{Co-instructor, Fundamentals of Accelerator Physics and Technology}{USPAS hosted by Michigan State University}{June 2018}{}{}

\cventry{Chicago, IL}{Adjunct, Physics of Sound and Light}{Vandercook School of Music}{Spring 2018}{}{}%}

\cventry{Lanzhou, China}{TA, Technical English}{Institute of Modern Physics}{July 2017}{}{}%}

\cventry{Beijing, China}{TA, Technical English}{Tsinghua University}{June 2015, 2016, 2017}{}{}%}

\cventry{New Brunswick, NJ}{TA, Fundamentals of Accelerator Physics and Technology}{USPAS hosted by Rutgers University}{June 2015}{}{}

\cventry{Chicago, IL}{TA, Electromagnetism and Optics \& Instrumentation Laboratory}{Illinois Institute of Technology}{2013--2015}{}{}

\section{Awards}
\cventry{}{Two million core hours awarded on Theta, for work done in collaboration with PSI.}{Dircetor's Discretionary Allocation}{ 2018}{}{}%}
\cventry{}{Funds granted to travel and present work at IPAC 2018 in Canada.}{IPAC 2018 Travel Award}{April 2018}{}{}%}
\cventry{}{Funds granted to travel and present work at IPAC 2017 in Denmark.}{IPAC 2017 Travel Award}{May 2017}{}{}%}
\cventry{}{Funds granted to attend a short course on beam dynamics at NAPAC 2016.}{Phelps Grant Award}{Nov 2016}{}{}%}
\cventry{}{One year grant funded by DOE; for TBA work done at ANL.}{Science Graduate Student Research Award}{2015--2016}{}{}%}

\section{Programming Languages}
\cvlistdoubleitem{\cvitem{Proficient in}{OPAL-T, Python, Matlab}}{\cvitem{Prior experience with}{C/C++, Fortran, Perl, SQL}}

\section{USPAS Courses Taken}
\begin{itemize}
	\item {Microwave Measurements and Beam Instrumentation}
	\item {Fundamentals of Timing and Synchronization}
	\item {Vacuum Science and Technology}
	\item {Accelerator Physics Using Maple}
\end{itemize}


\section{Recent Volunteer work}
\cventry{}{Visited Spry CLHS and Glenbard East High School.}{Hour of Code}{Dec 2017}{}{}
\cventry{}{Answered questions from high school students online.}{Adopt-a-Physicist}{Oct 2017}{}{}
\cventry{}{Monroe School}{Science Fair Judge}{Feb 2017}{}{}
\cventry{}{Hosted by ANL, participated as a mentor}{Introduce a girl to engineering day}{April 2017, Feb 2018}{}{}


\section{Memberships}
\cvlistdoubleitem{American Physical Society}{Society of Women Engineers}
\cvlistdoubleitem{Institute of Electrical and Electronics Engineers}{}

\iffalse
\section{References}
\begin{cvcolumns}
  \cvcolumn{Category 1}{\begin{itemize}\item Linda Spentzouris \item John Power \item Larry Pinsky \end{itemize}}
  \cvcolumn{Category 2}{Amongst others:\begin{itemize}\item Person 1, and\item Person 2\end{itemize}(more upon request)}
  \cvcolumn[0.5]{All the rest \& some more}{\textit{That} person, and \textbf{those} also (all available upon request).}
\end{cvcolumns}
\fi


\clearpage
%-----       letter       ---------------------------------------------------------
% recipient data
\iffalse
\recipient{USPAS}{Fermilab, MS 125\\Kirk Rd \& Wilson St.\\Batavia, IL  60510}
\date{August 15, 2017}
\opening{Dear Sir or Madam,}
\closing{Thank you,}
\enclosure[Attached]{curriculum vit\ae{}}          % use an optional argument to use a string other than "Enclosure", or redefine \enclname
\makelettertitle

I can not exaggerate the positive impact that USPAS has had on my graduate eduction and training, 
as each class provided crucial information that expanded my knowledge of accelerator physics. 
USPAS is unique in that it offers education in 
topics that can not be found at any university, and the lab intensive classes offer hands on training 
that is hard to come by. In addition to spectacular classes, every breakfast, lunch, homework session, 
and break is an opportunity to exchange ideas with scientist and engineers from labs across the country and the world.
These interactions, in combination with the course work, help me learn about what is going on in the field.
It can also be a time to learn some interesting history!

It is exciting to bring those ideas, new and old, back to my group at the 
Argonne Wakefield Accelerator (AWA) facility. As a PhD student, I started full time at the AWA 
after two years of classes. I then spent the next two years doing photoinjector 
simulations and experiments for the two electron beam lines at the AWA. 
I've also done UV optics work in order to improve the charge distribution in our bunch trains. 
The impact of USPAS classes in my work is tangible. The information and resources (the books are great) 
I acquired at USPAS have helped me find insight in my simulations, understand our timing and rf systems better, 
and appreciate the difficulty of maintaining a pristine vacuum as most facilities do. This information has helped me 
as I design a beam line for upcoming two beam acceleration experiments, as I work on our UV laser, and while
talking to vendors. USPAS has given me training in content and skills that are critical for anyone working
at an accelerator facility, and I hope to continue learning from this invaluable program.

\makeletterclosing
\fi 

%\clearpage\end{CJK*}                              % if you are typesetting your resume in Chinese using CJK; the \clearpage is required for fancyhdr to work correctly with CJK, though it kills the page numbering by making \lastpage undefined
\end{document}


%% end of file `template.tex'.
