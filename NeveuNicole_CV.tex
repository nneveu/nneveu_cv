%% start of file `template.tex'.
%% Copyright 2006-2013 Xavier Danaux (xdanaux@gmail.com).
%
% This work may be distributed and/or modified under the
% conditions of the LaTeX Project Public License version 1.3c,
% available at http://www.latex-project.org/lppl/.


\documentclass[11pt,a4paper,sans]{moderncv}        % possible options include font size ('10pt', '11pt' and '12pt'), paper size ('a4paper', 'letterpaper', 'a5paper', 'legalpaper', 'executivepaper' and 'landscape') and font family ('sans' and 'roman')

% moderncv themes
\moderncvstyle{banking}                            % style options are 'casual' (default), 'classic', 'oldstyle' and 'banking'
\moderncvcolor{blue}                                % color options 'blue' (default), 'orange', 'green', 'red', 'purple', 'grey' and 'black'
%\renewcommand{\familydefault}{\sfdefault}         % to set the default font; use '\sfdefault' for the default sans serif font, '\rmdefault' for the default roman one, or any tex font name
%\nopagenumbers{}                                  % uncomment to suppress automatic page numbering for CVs longer than one page

% character encoding
\usepackage[utf8]{inputenc} % if you are not using xelatex ou lualatex, replace by the encoding you are using
%\usepackage{hyperref}

% adjust the page margins
\usepackage[margin=0.7in]{geometry}
%\setlength{\hintscolumnwidth}{3cm}                % if you want to change the width of the column with the dates
%\setlength{\makecvtitlenamewidth}{10cm}           % for the 'classic' style, if you want to force the width allocated to your name and avoid line breaks. be careful though, the length is normally calculated to avoid any overlap with your personal info; use this at your own typographical risks...
                           
%\address{9700 Cass Avenue}{Lemont, IL 60439}{USA}
%\phone[fixed]{+1~(630)~252~7333}                               
%\email{nneveu@hawk.iit.edu}
%\homepage{www.johndoe.com}                         
%\quote{Some quote}                                 

% to show numerical labels in the bibliography (default is to show no labels); only useful if you make citations in your resume
%\makeatletter
%\renewcommand*{\bibliographyitemlabel}{\@biblabel{\arabic{enumiv}}}
%\makeatother
%\renewcommand*{\bibliographyitemlabel}{[\arabic{enumiv}]}% CONSIDER REPLACING THE ABOVE BY THIS

% bibliography with mutiple entries
\usepackage[resetlabels,labeled]{multibib}
\newcites{inprog}{Publications in Progress}
\newcites{refproc}{Refereed Proceedings}
\newcites{proc}{Proceedings}

%--------------------------------------------------------------------------------
\name{Nicole}{Neveu}
\extrainfo{PhD Candidate, Illinois Institute of Technology}

\begin{document}


\recipient{ATPESC Statement of Purpose}{}
\date{March 15, 2018}
\opening{To whom it may concern,}
\closing{Thank you,}
\enclosure[Attached]{curriculum vit\ae{}}          

\makelettertitle

As a graduate student in the Argonne Wakefield Accelerator group, 
I spend the majority of my time modeling complex and dynamic particle accelerators on 
Bebop, a HPC cluster provided by the Laboratory Computing Resource Center (LCRC) at ANL. 
Prior to Bebop, I was using it's predecessor Blues, and more recently, 
my colleagues and I were rewarded a 2 million core hour Director's Discretionary 
allocation to extend our work on Theta, 
an Argonne Leadership Computing Facility (ALCF) machine. 

In all of my work on these machines, I use the parallel, open source, and 
3D Particle-in-cell code OPAL-T. 
This code was chosen for several reasons, the most important being the ability 
to model space charge effects in 3D. At the AWA, high charge experiments are unique
in that space charge is a dominating nonlinear effect. This is a regime that most 
particle accelerator facilities avoid due to the detrimental effects on the beam.
However, there are a class of experiments that can benefit from high charge bunches 
if modeled and controlled with some degree of accuracy. This is the goal of my research
at the AWA, to understand and optimize machine parameters for 
this non-linear regime. This in combination with other higher order collective 
effects that machines everywhere must deal with, such as non-linearities in field maps of 
magnets or radio-frequency cavities, non-uniform distributions of the particles as they 
leave the source, and energy loss through bending elements which effect the trajectories downstream. 

There are two main types of simulations I carry out on these machines; 
low to mid fidelity large scale ensemble simulations for accelerator optimization,
and the mid to high fidelity simulations for comparison to experiments. 
Dimensionality in the optimization work can range from as little as 2 or 3 parameters 
to 10's of parameters, which creates a large search space. Most of this work is carried
out using OPAL-T in combination with parallel python code developed by Jeff Larson (ANL) 
and others in the library libEnsemble. These runs contain multiple parallel instances of OPAL-T initiated and running in the same batch job. Consider a 1,000 point random sample done for probing a 5-D parameter space; I start hundreds of low fidelity 8 core OPAL jobs on a large allocation.
As one OPAL simulation is finished (scale of few minutes) a new one is started.
In this way, we can create the large simulation data sets needed for surrogate models or optimization in a reasonable amount of time. 

The work on Theta focuses on high fidelity simulations, 
including scaling studies and preparations for a 1:1 simulations
 (40 billion particles) in collaboration with the UCLA plasma wakefield group. 
 In the case of plasma wakefield simulations, high fidelity simulations are 
 required to reach the level of statistics needed to resolve the plasma behavior.
 We would also like to optimize for KNL, so that the high fidelity simulations
 are more efficient. This requires updates to the FFT, particle push routine, and
 other aspects of the code. If reasonble scaling can be achieved this 
 ground work will be used as foundation for an INCITE proposal next year.
 
 While I have learn much about HPC systems and research through collaboration and 
 day to day research activities, as a Physics and first engineering student, 
 I would like the opportunity to learn about CS\&E research in a structured setting.
 I think I would benefit from the exposure to CS concepts from a CS point of view.


\makeletterclosing


\clearpage

\name{Nicole}{Neveu}
\title{CV}                               % optional, remove / comment the line if not wanted
\address{9700 Cass Avenue}{Lemont, IL 60439}{USA}% optional, remove / comment the line if not wanted; the "postcode city" and and "country" arguments can be omitted or provided empty
%\phone[mobile]{+1~(234)~567~890}                  
\phone[fixed]{+1~(630)~252~7333}                   
%\phone[fax]{+3~(456)~789~012}                                                  
\email{nneveu@hawk.iit.edu}
\extrainfo{nneveu@anl.gov}
\makecvtitle

\section{Education}
\cventry{Chicago, IL}{Illinois Institute of Technology}{Ph.D Candidate in Physics}{2013--Present}{College of Science}{}%{\textit{Grade}}{Description}  % arguments 3 to 6 can be left empty
\cventry{Houston, TX}{University of Houston}{B.S. Electrical Engineering}{2009--2013}{Cullen College of Engineering}{}

\iffalse
\section{Ph.D Thesis}
\cvitem{Title}{\emph{Design for Staged Two Beam Acceleration at the Argonne Wakefield Accelerator}}
\cvitem{Advisors}{Linda Spentzouris, John Power}
\cvitem{Summary}{
	Staged two beam acceleration using dielectric structures has yet to 
	be achieved anywhere in the world. In this thesis, I discuss beam 
	line design, simulation, and optimization, of a 
	beam line with the potential for dielectric two beam acceleration.
    Preliminary and prepratory experimental measurements
    are included.    }
\fi

\section{Research Experience}
\cventry{Lemont, IL}{Argonne National Laboratory (ANL)}{Graduate Student, Argonne Wakefield Accelerator (AWA)}{2013--Present}{}{
Design, optimization, and simulation of rf photoinjectors at AWA using \newline 
the accelerator physics PIC code, OPAL-T, and experimental validation of results. \newline Completed and ongoing work includes:
\begin{itemize}%
	\item Simulations:
	\begin{itemize}%
		\item Collaborating with Jeff Larson (ANL) on using novel optimization algorithms for accelerator physics.
		\item Large ensemble HPC simulation work (i.e. 10's of thousands of simulations per batch job).
		\item Collaborating with Andreas Adelmann (PSI) on using surrogate models for parameter space definition.
		\item High charge linac optimization (40 nC) using nlopt python package.
		\item Built and installed OPAL-T on two Linux clusters at ANL: Bebop and Theta.
		\item Wrote beam image processing script in python.
		\item Two beam acceleration (TBA) beam line design using OPAL-T.
		\item Optical Transistion Radiation (OTR) at 1nC.
		\item Thermionic RF gun, designed by Euclid TechLabs.
	\end{itemize}
	\item Experimental:
	\begin{itemize}%		
		\item Beam size, charge, and energy measurements
		\item RF power measurements for linac tanks and gun.
		\item 40nC linac emittance and bunch length measurements.
		\item Intensity improvement of UV laser pulse train 
		\item UV multisplitter assembly and characterization
		\item Design of relay imaging transport system for drive beam line.
	\end{itemize}
\end{itemize}
}
\section{Publications}
% Publications from a BibTeX file without multibib
%  for numerical labels: \renewcommand{\bibliographyitemlabel}{\@biblabel{\arabic{enumiv}}}% CONSIDER MERGING WITH PREAMBLE PART
%  to redefine the heading string ("Publications"): \renewcommand{\refname}{Articles}
%\nocite{*}
%\bibliographystyle{plain}
%\bibliography{publications}                        % 'publications' is the name of a BibTeX file

\nociteinprog{*}
\bibliographystyleinprog{ieeetr}
\bibliographyinprog{./cvparts/inprog}

\nociterefproc{*}
\bibliographystylerefproc{ieeetr}
\bibliographyrefproc{./cvparts/refproc}

%\nociteproc{*}
%\bibliographystyleproc{ieeetr}
%\bibliographyproc{./cvparts/proceedings}



\section{Research Presentations}

\subsection{Invited Conference or Workshop} %Workshops, Symposium
\textbf{Study of space charge dominated beams at the AWA rf photoinjector}\newline
Space Charge 2017, Darmstadt, Germany. October 6, 2017

\subsection{Seminar and Colloquia}
%\subsection{Other Talks}
\textbf{Photoinjector simulations using OPAL-t for the Argonne Wakefield Accelerator}\newline
Beam Dynamics Palaver AMAS, Paul Scherrer Institute, Villigen, Switzerland. May 23, 2017 
\vspace{0.3em}

\textbf{Research at the Argonne Wakefield Accelerator}\newline
Young Scientist Symposium, HEP ANL, Lemont, IL. May 17, 2016 
\vspace{0.3em}

\textbf{Two Beam Acceleration at the Argonne Wakefield Accelerator}\newline
Center for Accelerator and Particle Physics, IIT (CAPP), Chicago, IL. April 28, 2016 

\section{Awards}
\cventry{}{Two million core hours awarded on Theta, for work done in collaboration with PSI.}{ANL Dircetor's Discretionary Allocation}{ 2018}{}{}%}
\cventry{}{Funds granted to travel and present work at IPAC 2018 in Canada.}{IPAC 2018 Travel Award}{April 2018}{}{}%}
\cventry{}{Funds granted to travel and present work at IPAC 2017 in Denmark.}{IPAC 2017 Travel Award}{May 2017}{}{}%}
\cventry{}{Funds granted to attend a short course on beam dynamics at NAPAC 2016.}{Phelps Grant Award}{Nov 2016}{}{}%}
\cventry{}{One year grant funded by DOE; for TBA work done at ANL.}{Science Graduate Student Research Award}{2015--2016}{}{}%}

\section{Programming Languages}
\cvlistdoubleitem{\cvitem{Proficient in}{OPAL-T, Python, Matlab}}{\cvitem{Prior experience with}{C/C++, Fortran, Perl, SQL}}

\section{Teaching}
%\cventry{East Lansing, MI}{Co-instructor, Fundamentals of Accelerator Physics and Technology}{USPAS hosted by Michigan State University}{June 2018}{}{}

\cventry{Chicago, IL}{Adjunct, Physics of Sound and Light}{Vandercook School of Music}{Spring 2018}{}{}%}

\cventry{Lanzhou, China}{TA, Technical English}{Institute of Modern Physics}{July 2017}{}{}%}

\cventry{Beijing, China}{TA, Technical English}{Tsinghua University}{June 2015, 2016, 2017}{}{}%}

\cventry{New Brunswick, NJ}{TA, Fundamentals of Accelerator Physics and Technology}{USPAS hosted by Rutgers University}{June 2015}{}{}

\cventry{Chicago, IL}{TA, Electromagnetism and Optics \& Instrumentation Laboratory}{Illinois Institute of Technology}{2013--2015}{}{}

\iffalse
\section{USPAS Courses Taken}
\begin{itemize}
	\item {Microwave Measurements and Beam Instrumentation}
	\item {Fundamentals of Timing and Synchronization}
	\item {Vacuum Science and Technology}
	\item {Accelerator Physics Using Maple}
\end{itemize}
\fi

\section{Recent Volunteer work}
\cventry{}{Visited Spry CLHS and Glenbard East High School.}{Hour of Code}{Dec 2017}{}{}
\cventry{}{Answered questions from high school students online.}{Adopt-a-Physicist: \href{https://www.adoptaphysicist.org/webdocs/about.cfm}{Adopt}}{Oct 2017}{}{}
\cventry{}{Monroe School}{Science Fair Judge}{Feb 2017}{}{}
\cventry{}{Hosted by ANL, participated as a mentor}{Introduce a girl to engineering day}{April 2017, Feb 2018}{}{}


\section{Memberships}
\cvlistdoubleitem{American Physical Society}{Society of Women Engineers}
\cvlistdoubleitem{Institute of Electrical and Electronics Engineers}{}

\iffalse
\section{References}
\begin{cvcolumns}
  \cvcolumn{Category 1}{\begin{itemize}\item Linda Spentzouris \item John Power \item Larry Pinsky \end{itemize}}
  \cvcolumn{Category 2}{Amongst others:\begin{itemize}\item Person 1, and\item Person 2\end{itemize}(more upon request)}
  \cvcolumn[0.5]{All the rest \& some more}{\textit{That} person, and \textbf{those} also (all available upon request).}
\end{cvcolumns}
\fi


\clearpage


%\clearpage\end{CJK*}                              % if you are typesetting your resume in Chinese using CJK; the \clearpage is required for fancyhdr to work correctly with CJK, though it kills the page numbering by making \lastpage undefined
\end{document}


%% end of file `template.tex'.
