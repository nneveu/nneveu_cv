%% Copyright 2006-2013 Xavier Danaux (xdanaux@gmail.com).
%
% This work may be distributed and/or modified under the
% conditions of the LaTeX Project Public License version 1.3c,
% available at http://www.latex-project.org/lppl/.

\documentclass[11pt,a4paper,sans]{moderncv}        % possible options include font size ('10pt', '11pt' and '12pt'), paper size ('a4paper', 'letterpaper', 'a5paper', 'legalpaper', 'executivepaper' and 'landscape') and font family ('sans' and 'roman')

% moderncv themes
\moderncvstyle{banking}         % style options are 'casual' (default), 'classic', 'oldstyle' and 'banking'
\moderncvcolor{blue}          % color options 'blue' (default), 'orange', 'green', 'red', 'purple', 'grey' and 'black'
%\renewcommand{\familydefault}{\sfdefault}         % to set the default font; use '\sfdefault' for the default sans serif font, '\rmdefault' for the default roman one, or any tex font name
%\nopagenumbers{}                                  % uncomment to suppress automatic page numbering for CVs longer than one page

% character encoding
\usepackage[utf8]{inputenc} % if you are not using xelatex ou lualatex, replace by the encoding you are using
%\usepackage{hyperref}
\usepackage{multicol}
\usepackage{url}
% adjust the page margins
\usepackage[margin=0.7in]{geometry}
%\setlength{\hintscolumnwidth}{3cm}                % if you want to change the width of the column with the dates
%\setlength{\makecvtitlenamewidth}{10cm}           % for the 'classic' style, if you want to force the width allocated to your name and avoid line breaks. be careful though, the length is normally calculated to avoid any overlap with your personal info; use this at your own typographical risks...

% to show numerical labels in the bibliography (default is to show no labels); only useful if you make citations in your resume
%\makeatletter
%\renewcommand*{\bibliographyitemlabel}{\@biblabel{\arabic{enumiv}}}
%\makeatother
%\renewcommand*{\bibliographyitemlabel}{[\arabic{enumiv}]}% CONSIDER REPLACING THE ABOVE BY THIS

% bibliography with mutiple entries
\usepackage[resetlabels,labeled]{multibib}
\newcites{art}{Journal}
\newcites{refproc}{Technical Notes \& Refereed Proceedings }
\newcites{inprog}{Publications in Preparation}
\newcites{proc}{Select Conference Proceedings}

%--------------------------------------------------------------------------------
\name{Nicole}{Neveu}
\extrainfo{nneveu@slac.stanford.edu}

\begin{document}
%-----       letter       ---------------------------------------------------------
% recipient data
%\title{Cover Letter}
%\recipient{}{SLAC National Accelrator Laboratory\\2575 Sandhill Rd. \\ Menlo Park, CA}
%\date{November 15, 2019}
%\opening{Dear Committee Members,}
%\closing{\vspace{1em} \\Sincerly,\vspace{4em}}
%\enclosure[Attached]{curriculum vit\ae{}}
%\makelettertitle
%
%\input{cover_letters/}
%
%
%\makeletterclosing

%\clearpage

\name{Nicole}{Neveu}
\title{CV}  % optional, remove / comment the line if not wanted
\address{2575 Sandhill Road}{Menlo Park, CA 94025}{USA}% optional, remove the line if not wanted; the "postcode city"and "country" arguments can be omitted or provided empty
\phone[fixed]{+1~(832)~628-4671}
\email{nneveu@slac.stanford.edu}
\extrainfo{nneveu@stanford.edu}
\makecvtitle

\section{Education}
\cventry{Chicago, IL}{Illinois Institute of Technology}{Ph.D Physics}{2013--2018}{College of Science}{}%{\textit{Grade}}{Description}  % arguments 3 to 6 can be left empty
\cventry{Houston, TX}{University of Houston}{B.S. Electrical Engineering}{2009--2013}{Cullen College of Engineering}{}

\iffalse
\section{Ph.D Thesis}
\cvitem{Title}{\emph{Design for  Staged Two Beam Acceleration at the Argonne Wakefield Accelerator}}
\cvitem{Advisors}{Linda Spentzouris, John Power}
\cvitem{Summary}{
	I discuss beam line design, simulation, and optimization, of a
	beam line with the potential for dielectric two beam acceleration.
    Preliminary and prepratory experimental measurements
    are included.    }
\fi

\section{Research Experience}
\cventry{Menlo Park, CA}{SLAC National Accelerator Laboratory}{Research Associate -- Experimental}{December 2018--Present}{}{
Beam dynamics work focused on LCLS/LCLS-II includes:
\begin{itemize}
	\item Experimental
	\begin{itemize}
         \item Submitted proposal to Laser Improvement Task force for new termporal laser diagnostic. This will provide a direct measurment of the laser's longitudinal profile.
		\item Extended emittance measurement GUI for LCLS-II
		\item Participated in LCLS-II gun commissioning: emittance and beam size measurements, QE maps, etc.
        \item Took beam measurements on LCLS injector and LCLS-II gun for machine learning models and benchmarking.
        \item Contributing to and managing repository for general data processing Python tools for LCLS:
        \begin{itemize}
            \item \url{https://github.com/slaclab/lcls-tools}
        \end{itemize}
    \item Helping with experimental runs at the X-band test facility in TID
	\end{itemize}
	\item Simulations
	\begin{itemize}
        \item Working with group members on uncertainty quantification analysis of LCLS-II
        \item Study and optimization of Sum Frequency Generated laser pulses in LCLS-II injector
        \item Optimization of beam dynamics for Gaussian temporal profiles in LCLS-II commissioning planning
		\item Incorporation of transverse laser profiles in simulation using Virtual Cathode Camera images
        \item Working with group members to inform surrogate models (machine learning models)
        \item Creating simulation and measured data sets for machine learning studies
        \item Improving PIC simulation models for LCLS/LCLS-II
        \item Simulation of the X-band test facility in TID to support re-start of facility
	\end{itemize}
\end{itemize}
}

\vspace{1em}
\cventry{Lemont, IL}{Argonne National Laboratory (ANL)}{Graduate Student, Argonne Wakefield Accelerator (AWA)}{2013--2018}{}{
Optimization, and simulation of rf photoinjectors at AWA using
the parallel PIC code, OPAL-T, \newline and experimental validation of results. Completed work includes:
\begin{itemize}%
	\item Thesis: Design for Staged Two Beam Acceleration (TBA) at the Argonne Wakefield Accelerator
	\begin{itemize}
		\item Simulations and experimental preparations for implementation of staged TBA at AWA.
	\end{itemize}
	\item Simulations:
	\begin{itemize}%
		\item TBA beam line design at 40 nC for AWA.
		\item Large ensemble HPC simulations.
		\item High charge linac optimization (40 nC) using nlopt python package.
		\item Optical Transition Radiation (OTR) simulations at 1 nC.
		\item Simulation of the Euclid Techlabs thermionic gun design.
		\item Wrote post processing script in python for analyzing transverse beam images.
		\item Built and installed OPAL-T on two Linux clusters at ANL: Bebop and Theta.
		%\item \url{https://github.com/nneveu}
	\end{itemize}
	\item Experimental:
	\begin{itemize}%
		\item Testing of high charge fast rise time kicker
		\item Beam size, charge, and energy measurements.
		\item RF power measurements for linac tanks and gun.
		\item 40 nC emittance and bunch length measurements.
		\item Intensity improvement of UV laser pulse train.
		\item UV multisplitter assembly and characterization.
		\item Design of relay imaging transport system for drive beam line.
	\end{itemize}
\end{itemize}
}

\newpage

\section{Publications}
% Publications from a BibTeX file without multibib
%  for numerical labels: \renewcommand{\bibliographyitemlabel}{\@biblabel{\arabic{enumiv}}}% CONSIDER MERGING WITH PREAMBLE PART
%  to redefine the heading string ("Publications"): \renewcommand{\refname}{Articles}

\nociteart{*}
\bibliographystyleart{ieeetr}
\bibliographyart{./cvparts/papers}
\vspace{0.5em}

\nociterefproc{*}
\bibliographystylerefproc{ieeetr}
\bibliographyrefproc{./cvparts/refproc}
\vspace{0.5em}

\nociteinprog{*}
\bibliographystyleinprog{ieeetr}
\bibliographyinprog{./cvparts/inprog}
\vspace{0.5em}


%\section{Proceedings}
\nociteproc{*}
\bibliographystyleproc{ieeetr}
\bibliographyproc{./cvparts/proceedings}
\vspace{0.5em}
%
%\newpage

\section{Research Presentations}
\subsection{Conference or Workshop} %Workshops, Symposium
\textbf{Nonlinear Shaped Laser Pulses in the LCLS-II Photoinjector }\newline
CLEO, Virtual, Scheduled for May 2021
\vspace{0.3em}

\textbf{Benchmarking the LCLS-II Photoinjector}\newline
NAPAC 2019, East Lansing, MI. September 2019
\vspace{0.3em}

\textbf{Surrogate Modeling for Charged Particle Accelerator Beam Dynamics}\newline
HZB, Berlin, Germany. April 4, 2019
\vspace{0.3em}

\textbf{Comparison of Model Based and Heuristic Optimization Algorithms \newline
	Applied to Photoinjectors Using Libensemble}\newline
International Computational Accelertor Physics 2018, Key West, FL. October 2018
\vspace{0.3em}

\textbf{Study of space charge dominated beams at the AWA rf photoinjector}\newline
Space Charge 2017, Darmstadt, Germany. October 6, 2017
\vspace{0.3em}

\subsection{Seminar and Colloquia}
%\subsection{Other Talks}

%\textbf{LCLS-II Optimization}\newline
%SLAC ARD Seminar, Virtual, December, 2020
%\vspace{0.3em}

\textbf{Photoinjectors: from wakefields to free electron lasers,\\ and the intersection of math and accelerator physics}\newline
University of Hawai'i at Manoa, Honolulu, Hawai'i. March 9, 2020
\vspace{0.3em}

\textbf{Modeling the LCLS-II Injector: from Optimization to Machine Learning}\newline
Lawrence Berkeley National Laboratory, Berkeley, California. December 6, 2019
\vspace{0.3em}

%\textbf{Emittance GUI Studies}\newline
%SLAC National Accelerator Laboratory, Menlo Park, CA. September 26, 2019
%\vspace{0.3em}

\textbf{Fully Staged Two Beam Acceleration at the Argonne Wakefield Accelerator}\newline
MIT, Boston, MA. August 20, 2018
\vspace{0.3em}

\textbf{Fully Staged Two Beam Acceleration at the Argonne Wakefield Accelerator}\newline
SLAC National Accelerator Laboratory, Menlo Park, CA. July 26, 2018
\vspace{0.3em}

\textbf{Fully Staged Two Beam Acceleration at the Argonne Wakefield Accelerator}\newline
Los Alamos National Laboratory, Los Alamos, NM. July 24, 2018
\vspace{0.3em}

\textbf{Simulations, Optimization, and Experimental Measurements at the AWA}\newline
RadiaSoft, Boulder, CO. June 29, 2018
\vspace{0.3em}

\textbf{Photoinjector simulations using OPAL-T for the Argonne Wakefield Accelerator}\newline
Beam Dynamics Palaver AMAS, Paul Scherrer Institute, Villigen, Switzerland. May 23, 2017
\vspace{0.3em}

\textbf{Research at the Argonne Wakefield Accelerator}\newline
Young Scientist Symposium, HEP ANL, Lemont, IL. May 17, 2016
\vspace{0.3em}

\textbf{Two Beam Acceleration at the Argonne Wakefield Accelerator}\newline
Center for Accelerator and Particle Physics, IIT (CAPP), Chicago, IL. April 28, 2016


%\newpage
\section{Awards}
\cventry{}{Attended summer of 2018.}{Argonne Training Program on Extreme Scale Computing (ATPESC)}{August 2018}{}{}
\cventry{}{Funds granted to travel and present work at AAC 2018.}{AAC 2018 Student Grant}{August 2018}{}{}

\cventry{}{Two million core hours awarded on Theta, for work done in collaboration with PSI.}{ANL Dircetor's Discretionary Allocation}{ 2018}{}{}%}
\cventry{}{Funds granted to travel and present work at IPAC 2018 in Canada.}{IPAC 2018 Student Travel Award}{April 2018}{}{}%}
\cventry{}{Funds granted to travel and present work at IPAC 2017 in Denmark.}{IPAC 2017 Student Travel Award}{May 2017}{}{}%}
\cventry{}{Funds granted to attend a short course on beam dynamics at NAPAC 2016.}{Phelps Grant Award}{Nov 2016}{}{}%}
\cventry{}{One year grant funded by DOE; for TBA work done at ANL.}{Science Graduate Student Research Award}{2015--2016}{}{}%}

\cventry{}{Attended the following classes:}{USPAS Scholarships}{2014-2018}{}{}
%\section{USPAS Courses Taken}
\begin{itemize}
	\item {Microwave Measurements and Beam Instrumentation, Fundamentals of Timing and Synchronization}
	\item {Vacuum Science and Technology, Accelerator Physics Using Maple}
\end{itemize}




%\newpage
\section{Teaching}

\cventry{Virtual}{Instructor, UHV Free Electron Lasers }{USPAS hosted by Texas A\&M}{January 2021}{}{}

\cventry{Albuquerque, NM}{Lab Instructor, Fundamentals of Accelerator Physics and Technology}{USPAS hosted by University of New Mexico}{June 2019}{}{}

\cventry{East Lansing, MI}{TA \& Guest Lecturer, Fundamentals of Accelerator Physics and Technology}{USPAS hosted by Michigan State University}{June 2018}{}{}

\cventry{Chicago, IL}{Adjunct, Physics of Sound and Light}{Vandercook School of Music}{Spring 2018}{}{}%}

\cventry{Lanzhou, China}{TA, Technical English}{Institute of Modern Physics}{July 2017}{}{}%}

\cventry{Beijing, China}{TA, Technical English}{Tsinghua University}{June 2015, 2016, 2017}{}{}%}

\cventry{New Brunswick, NJ}{TA, Fundamentals of Accelerator Physics and Technology}{USPAS hosted by Rutgers University}{June 2015}{}{}

\cventry{Chicago, IL}{TA, Electromagnetism and Optics \& Instrumentation Laboratory}{Illinois Institute of Technology}{2013--2015}{}{}

\section{Mentorship}
\cvitemwithcomment{Paris Franz}{Graduate student, Stanford University}{2020-Present}
\cvitemwithcomment{Lipi Gupta}{Graduate student, University of Chicago}{2019-Present}
\cvitemwithcomment{Mirian Juan}{Graduate student, CSU Long Beach}{2019-Present}
\cvitemwithcomment{Lucas Berens}{SULI Summer student, Optimization of LCLS-II injector using VCC images}{2020}
\cvitemwithcomment{Amy Sihn}{SULI Summer student, A Python emittance GUI for LCLS}{2020}
\cvitemwithcomment{McCay Rhodebeck}{CCI Summer student, lcls-tools Development in Python}{2020}



\newpage
\iftrue
\section{Recent Volunteer work}
\cventry{Virtual}{Introduction to Particle Accelerators}{Cabrillo High School STEM Talk}{Feb 2021}{}{}
\cventry{}{Hosted by SLAC, participated as volunteer}{SAGE-S Summer Camp for Girls}{August 2019, 2020}{}{}
\cventry{}{Hosted by SLAC, participated as volunteer}{Science Bowl}{January 2020}{}{}
\cventry{}{Answered questions from high school students online.}{Adopt-a-Physicist: \url{}{}}{Oct 2017, 2018, 2020}{}{\url{https://www.adoptaphysicist.org/webdocs/about.cfm}}
\cventry{}{Hosted by SLAC, participated as tour guide}{Community Day}{October 2019}{}{}

\cventry{}{Hosted by SLAC, participated as volunteer}{CORE Summer Science Institute}{June 2019}{}{}

\cventry{}{Hosted by ANL, participated as a mentor}{Introduce a girl to engineering day}{April 2017, Feb 2018}{}{}
\cventry{}{Visited Spry CLHS and Glenbard East High School.}{Hour of Code}{Dec 2017}{}{}
%\cventry{}{Monroe School}{Science Fair Judge}{Feb 2017}{}{}


\section{Programming Languages}
\cvlistdoubleitem{\cvitem{Proficient in}{ASTRA, OPAL-T, Python, Matlab}}{\cvitem{Prior experience with}{IMPACT, Genesis, C/C++}}

\section{Memberships}
\cvlistdoubleitem{American Physical Society}{Society of Women Engineers}
\cvlistdoubleitem{Institute of Electrical and Electronics Engineers}{}


\fi

%\iffalse
%\section{References}
%\vspace{1em}
%
%\begin{enumerate}
%	\itemsep1em
%	\item Linda Spentzouris, IIT
%	\begin{itemize}
%		\item[] Phone: 312.567.3577
%		\item[] Email: linda@agni.phys.iit.edu
%	\end{itemize}
%	\item Andreas Adelmann, PSI
%	\begin{itemize}
%		\item[] Phone: 41.56.310.4233
%		\item[] Email: andreas.adelmann@psi.ch
%	\end{itemize}
%	\item Jeffrey Larson, MCS-ANL
%	\begin{itemize}
%		\item[] Phone: 630.252.3221
%		\item[] Email: jmlarson@anl.gov
%	\end{itemize}
%	\item Chunguang Jing, Euclid Techlabs, LLC
%	\begin{itemize}
%		\item[] Phone: 630.313.6562
%		\item[] Email: jingchunguang@gmail.com
%	\end{itemize}
%
%\end{enumerate}
%\fi
%
%\clearpage




\end{document}
