%% start of file `template.tex'.
%% Copyright 2006-2013 Xavier Danaux (xdanaux@gmail.com).
%
% This work may be distributed and/or modified under the
% conditions of the LaTeX Project Public License version 1.3c,
% available at http://www.latex-project.org/lppl/.


\documentclass[11pt,a4paper,sans]{moderncv}        % possible options include font size ('10pt', '11pt' and '12pt'), paper size ('a4paper', 'letterpaper', 'a5paper', 'legalpaper', 'executivepaper' and 'landscape') and font family ('sans' and 'roman')

% moderncv themes
\moderncvstyle{banking}         % style options are 'casual' (default), 'classic', 'oldstyle' and 'banking'
\moderncvcolor{blue}          % color options 'blue' (default), 'orange', 'green', 'red', 'purple', 'grey' and 'black'
%\renewcommand{\familydefault}{\sfdefault}         % to set the default font; use '\sfdefault' for the default sans serif font, '\rmdefault' for the default roman one, or any tex font name
%\nopagenumbers{}                                  % uncomment to suppress automatic page numbering for CVs longer than one page

% character encoding
\usepackage[utf8]{inputenc} % if you are not using xelatex ou lualatex, replace by the encoding you are using
%\usepackage{hyperref}

% adjust the page margins
\usepackage[margin=0.7in]{geometry}
%\setlength{\hintscolumnwidth}{3cm}                % if you want to change the width of the column with the dates
%\setlength{\makecvtitlenamewidth}{10cm}           % for the 'classic' style, if you want to force the width allocated to your name and avoid line breaks. be careful though, the length is normally calculated to avoid any overlap with your personal info; use this at your own typographical risks...
                           
%\address{9700 Cass Avenue}{Lemont, IL 60439}{USA}
%\phone[fixed]{+1~(630)~252~7333}                               
%\email{nneveu@hawk.iit.edu}
%\homepage{www.johndoe.com}                         
%\quote{Some quote}                                 

% to show numerical labels in the bibliography (default is to show no labels); only useful if you make citations in your resume
%\makeatletter
%\renewcommand*{\bibliographyitemlabel}{\@biblabel{\arabic{enumiv}}}
%\makeatother
%\renewcommand*{\bibliographyitemlabel}{[\arabic{enumiv}]}% CONSIDER REPLACING THE ABOVE BY THIS

% bibliography with mutiple entries
\usepackage[resetlabels,labeled]{multibib}
\newcites{inprog}{Publications in Progress}
\newcites{refproc}{Refereed Proceedings}
\newcites{proc}{Proceedings}

%--------------------------------------------------------------------------------
\name{Nicole}{Neveu}
\extrainfo{PhD Candidate in Physics, Illinois Institute of Technology \& ANL}

\begin{document}

\iffalse
\recipient{ATPESC Statement of Purpose}{}
\date{March 15, 2018}
\opening{}
\closing{Thanks for your consideration,}
\enclosure[Attached]{curriculum vit\ae{}}          
\makelettertitle
\vspace{-4em}

As a 5th year graduate student in the Argonne Wakefield Accelerator (AWA) group,
I have learned most of what I know about HPC systems through collaboration and
day to day research activities. I would enjoy the opportunity to learn about HPC in a structured setting, as several of the topics covered relate to my current or upcoming research goals, 
especially numerical methods and performance analysis techniques. 
I spend the majority of my time modeling  
dynamic particle accelerators on Bebop, a HPC cluster provided by the 
LCRC at ANL. Prior to Bebop, I was using it's predecessor Blues, and more recently,
my colleagues and I were awarded a Director's Discretionary
allocation to extend our work on Theta,
an ALCF machine.
This course could help me learn how to use these machines more efficiently 
and prepare for the move exascale in the future.
Another topic of interest for me is "building community codes for HPC". 
In all of my work, I use the parallel, open source, and
3D particle-in-cell code OPAL-T. The developers have been great collaborators, 
and I would like to continue contributing to open source projects in 
a sustainable and forward thinking way.

Aside from being open source and parallel, I also use OPAL-T for it's ability to model space charge effects in 3D.
At the AWA, high charge experiments are unique
in that space charge is the dominating nonlinear effect. This is a regime that most
particle accelerator facilities avoid due to the detrimental effects on the beam
and difficulty of control.
However, there is a class of experiments that can benefit from high charge bunches
if modeled and controlled with some degree of accuracy. The goal of my research
at the AWA is to understand and optimize input parameters to accelerators for
this nonlinear regime. This is combined with other higher order effects 
such as asymmetries in the field maps of magnets or radio-frequency cavities, 
non-uniform distributions of the particles as they leave the source, 
and energy loss through bending elements.

With this in mind, the work I do on Bebop focuses on large scale ensemble optimization of the high charge accelerators I described above.
Dimensionality can range from three to about thirty parameters, 
for the AWA. For larger machines, the number of parameters could be in the 
thousands. I look for ways to optimize efficiently in these large search spaces.
I use parallel python code developed by Jeff Larson (ANL) and others in the library libensemble in combination with generating functions that I write.
These runs contain multiple parallel instances of OPAL-T initiated and running in the same batch job.
For a typical 1,000 point random sample done to probe a 7D parameter space,
I start hundreds of low fidelity eight core OPAL-T jobs on a large allocation.
As one simulation is finished (time scale of few minutes) a new one is started.
In this way, we can create the large data sets needed for surrogate models
or optimization in a reasonable amount of time. 
A third order surrogate model for a 7D parameter space requires ~16,000 simulations. 
This was accomplished in less than eight hours using the ensemble method. I also take advantage of the
built in genetic algorithm (GA) and parallel framework that is incorporated into OPAL.
These low to mid fidelity results can then be used to narrow the search space and 
guide where high fidelity simulations are needed.
The parameters also give guidance to the accelerator operators 
so that less time is spent hand tuning the machine.

Our work on Theta focuses on the high fidelity simulations,
including scaling studies and preparations for realistic, forty billion particle simulations in collaboration with the UCLA plasma wakefield group.
In this case, high fidelity simulations are required to resolve the plasma behavior.
We would also like to optimize for KNL. This will requires updates to the FFT, particle push routine, 
and several aspects of the code. 

In conclusion, based on the description of ATPESC, attendance could greatly 
impact my current/future research goals and help me become a more efficient HPC user.


\makeletterclosing

\clearpage
\fi

\name{Nicole}{Neveu}
\title{CV}                               % optional, remove / comment the line if not wanted
\address{9700 Cass Avenue}{Lemont, IL 60439}{USA}% optional, remove / comment the line if not wanted; the "postcode city" and and "country" arguments can be omitted or provided empty
%\phone[mobile]{+1~(234)~567~890}                  
\phone[fixed]{+1~(630)~252~7333}                   
%\phone[fax]{+3~(456)~789~012}                                                  
\email{nneveu@hawk.iit.edu}
\extrainfo{nneveu@anl.gov}
\makecvtitle

\section{Education}
\cventry{Chicago, IL}{Illinois Institute of Technology}{Ph.D Candidate in Physics}{2013--Present}{College of Science}{}%{\textit{Grade}}{Description}  % arguments 3 to 6 can be left empty
\cventry{Houston, TX}{University of Houston}{B.S. Electrical Engineering}{2009--2013}{Cullen College of Engineering}{}

\iffalse
\section{Ph.D Thesis}
\cvitem{Title}{\emph{Design for Staged Two Beam Acceleration at the Argonne Wakefield Accelerator}}
\cvitem{Advisors}{Linda Spentzouris, John Power}
\cvitem{Summary}{
	Staged two beam acceleration using dielectric structures has yet to 
	be achieved anywhere in the world. In this thesis, I discuss beam 
	line design, simulation, and optimization, of a 
	beam line with the potential for dielectric two beam acceleration.
    Preliminary and prepratory experimental measurements
    are included.    }
\fi

\section{Research Experience}
\cventry{Lemont, IL}{Argonne National Laboratory (ANL)}{Graduate Student, Argonne Wakefield Accelerator (AWA)}{2013--Present}{}{
Design, optimization, and simulation of rf photoinjectors at AWA using \newline 
the parallel PIC code, OPAL-T, and experimental validation of results. \newline Completed and ongoing work includes:
\begin{itemize}%
	\item Simulations:
	\begin{itemize}%
		\item Two beam acceleration (TBA) beam line design at 40 nC for AWA (in progress).
		\item Collaborating with Auralee Edelen and Andreas Adelmann (PSI) on using surrogate models for accelerator simulations.
		\item Collaborating with Jeff Larson (ANL) on using novel optimization algorithms for accelerators. 
		\item Large ensemble HPC simulations (for work listed above).
		\item High charge linac optimization (40 nC) using nlopt python package.
		\item Optical Transistion Radiation (OTR) simulations at 1 nC.
		\item Thermionic RF gun simulations, designed by Euclid TechLabs.
		\item Wrote post processing script in python for transverse beam images.
		\item Built and installed OPAL-T on two Linux clusters at ANL: Bebop and Theta.
	\end{itemize}
	\item Experimental:
	\begin{itemize}%		
		\item Beam size, charge, and energy measurements.
		\item RF power measurements for linac tanks and gun.
		\item 40 nC emittance and bunch length measurements.
		\item Intensity improvement of UV laser pulse train. 
		\item UV multisplitter assembly and characterization.
		\item Design of relay imaging transport system for drive beam line.
	\end{itemize}
\end{itemize}
}
\section{Publications}
% Publications from a BibTeX file without multibib
%  for numerical labels: \renewcommand{\bibliographyitemlabel}{\@biblabel{\arabic{enumiv}}}% CONSIDER MERGING WITH PREAMBLE PART
%  to redefine the heading string ("Publications"): \renewcommand{\refname}{Articles}
%\nocite{*}
%\bibliographystyle{plain}
%\bibliography{publications}                        % 'publications' is the name of a BibTeX file

\nociteinprog{*}
\bibliographystyleinprog{ieeetr}
\bibliographyinprog{./cvparts/inprog}

\nociterefproc{*}
\bibliographystylerefproc{ieeetr}
\bibliographyrefproc{./cvparts/refproc}

\nociteproc{*}
\bibliographystyleproc{ieeetr}
\bibliographyproc{./cvparts/proceedings}



\section{Research Presentations}

\subsection{Invited Conference or Workshop} %Workshops, Symposium
\textbf{Study of space charge dominated beams at the AWA rf photoinjector}\newline
Space Charge 2017, Darmstadt, Germany. October 6, 2017

\subsection{Seminar and Colloquia}
%\subsection{Other Talks}
\textbf{Photoinjector simulations using OPAL-T for the Argonne Wakefield Accelerator}\newline
Beam Dynamics Palaver AMAS, Paul Scherrer Institute, Villigen, Switzerland. May 23, 2017 
\vspace{0.3em}

\textbf{Research at the Argonne Wakefield Accelerator}\newline
Young Scientist Symposium, HEP ANL, Lemont, IL. May 17, 2016 
\vspace{0.3em}

\textbf{Two Beam Acceleration at the Argonne Wakefield Accelerator}\newline
Center for Accelerator and Particle Physics, IIT (CAPP), Chicago, IL. April 28, 2016 

\section{Awards}
\cventry{}{Two million core hours awarded on Theta, for work done in collaboration with PSI.}{ANL Dircetor's Discretionary Allocation}{ 2018}{}{}%}
\cventry{}{Funds granted to travel and present work at IPAC 2018 in Canada.}{IPAC 2018 Travel Award}{April 2018}{}{}%}
\cventry{}{Funds granted to travel and present work at IPAC 2017 in Denmark.}{IPAC 2017 Travel Award}{May 2017}{}{}%}
\cventry{}{Funds granted to attend a short course on beam dynamics at NAPAC 2016.}{Phelps Grant Award}{Nov 2016}{}{}%}
\cventry{}{One year grant funded by DOE; for TBA work done at ANL.}{Science Graduate Student Research Award}{2015--2016}{}{}%}

\section{Programming Languages}
\cvlistdoubleitem{\cvitem{Proficient in}{OPAL-T, Python, Matlab}}{\cvitem{Prior experience with}{C/C++, Fortran, Perl, SQL}}

\section{Teaching}
%\cventry{East Lansing, MI}{Co-instructor, Fundamentals of Accelerator Physics and Technology}{USPAS hosted by Michigan State University}{June 2018}{}{}

\cventry{Chicago, IL}{Adjunct, Physics of Sound and Light}{Vandercook School of Music}{Spring 2018}{}{}%}

\cventry{Lanzhou, China}{TA, Technical English}{Institute of Modern Physics}{July 2017}{}{}%}

\cventry{Beijing, China}{TA, Technical English}{Tsinghua University}{June 2015, 2016, 2017}{}{}%}

\cventry{New Brunswick, NJ}{TA, Fundamentals of Accelerator Physics and Technology}{USPAS hosted by Rutgers University}{June 2015}{}{}

\cventry{Chicago, IL}{TA, Electromagnetism and Optics \& Instrumentation Laboratory}{Illinois Institute of Technology}{2013--2015}{}{}

\iffalse
\section{USPAS Courses Taken}
\begin{itemize}
	\item {Microwave Measurements and Beam Instrumentation}
	\item {Fundamentals of Timing and Synchronization}
	\item {Vacuum Science and Technology}
	\item {Accelerator Physics Using Maple}
\end{itemize}
\fi

\section{Recent Volunteer work}
\cventry{}{Hosted by ANL, participated as a mentor}{Introduce a girl to engineering day}{April 2017, Feb 2018}{}{}
\cventry{}{Visited Spry CLHS and Glenbard East High School.}{Hour of Code}{Dec 2017}{}{}
\cventry{}{Answered questions from high school students online.}{Adopt-a-Physicist: \url{https://www.adoptaphysicist.org/webdocs/about.cfm}{}}{Oct 2017}{}{}
\cventry{}{Monroe School}{Science Fair Judge}{Feb 2017}{}{}



\section{Memberships}
\cvlistdoubleitem{American Physical Society}{Society of Women Engineers}
\cvlistdoubleitem{Institute of Electrical and Electronics Engineers}{}

\iffalse
\section{References}
\begin{cvcolumns}
  \cvcolumn{Category 1}{\begin{itemize}\item Linda Spentzouris \item John Power \item Larry Pinsky \end{itemize}}
  \cvcolumn{Category 2}{Amongst others:\begin{itemize}\item Person 1, and\item Person 2\end{itemize}(more upon request)}
  \cvcolumn[0.5]{All the rest \& some more}{\textit{That} person, and \textbf{those} also (all available upon request).}
\end{cvcolumns}
\fi


\clearpage


%\clearpage\end{CJK*}                              % if you are typesetting your resume in Chinese using CJK; the \clearpage is required for fancyhdr to work correctly with CJK, though it kills the page numbering by making \lastpage undefined
\end{document}


%% end of file `template.tex'.
