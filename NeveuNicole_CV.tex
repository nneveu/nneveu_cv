%% Copyright 2006-2013 Xavier Danaux (xdanaux@gmail.com).
%
% This work may be distributed and/or modified under the
% conditions of the LaTeX Project Public License version 1.3c,
% available at http://www.latex-project.org/lppl/.

\documentclass[11pt,a4paper,sans]{moderncv}        % possible options include font size ('10pt', '11pt' and '12pt'), paper size ('a4paper', 'letterpaper', 'a5paper', 'legalpaper', 'executivepaper' and 'landscape') and font family ('sans' and 'roman')

% moderncv themes
\moderncvstyle{banking}         % style options are 'casual' (default), 'classic', 'oldstyle' and 'banking'
\moderncvcolor{blue}          % color options 'blue' (default), 'orange', 'green', 'red', 'purple', 'grey' and 'black'
%\renewcommand{\familydefault}{\sfdefault}         % to set the default font; use '\sfdefault' for the default sans serif font, '\rmdefault' for the default roman one, or any tex font name
%\nopagenumbers{}                                  % uncomment to suppress automatic page numbering for CVs longer than one page

% character encoding
\usepackage[utf8]{inputenc} % if you are not using xelatex ou lualatex, replace by the encoding you are using
\usepackage{multicol}
%\usepackage{url}
% adjust the page margins
\usepackage[margin=0.75in]{geometry}
%\setlength{\hintscolumnwidth}{3cm}                % if you want to change the width of the column with the dates
%\setlength{\makecvtitlenamewidth}{10cm}           % for the 'classic' style, if you want to force the width allocated to your name and avoid line breaks. be careful though, the length is normally calculated to avoid any overlap with your personal info; use this at your own typographical risks...


% bibliography with mutiple entries
\usepackage[resetlabels,labeled]{multibib}
\newcites{art}{Journal}
\newcites{refproc}{Technical Notes \& Refereed Proceedings }
\newcites{inprog}{Publications in Preparation}
\newcites{proc}{Select Conference Proceedings}

%--------------------------------------------------------------------------------
\name{Nicole}{Neveu}
\extrainfo{nneveu@slac.stanford.edu}

\begin{document}
%-----       letter       ---------------------------------------------------------
% recipient data
%\title{Cover Letter}
%\recipient{}{SLAC National Accelrator Laboratory\\2575 Sandhill Rd. \\ Menlo Park, CA}
%\date{November 15, 2019}
%\opening{Dear Committee Members,}
%\closing{\vspace{1em} \\Sincerly,\vspace{4em}}
%\enclosure[Attached]{curriculum vit\ae{}}
%\makelettertitle
%
%\input{cover_letters/}
%
%
%\makeletterclosing

%\clearpage

\name{Nicole}{Neveu}
\title{CV}  % optional, remove / comment the line if not wanted
\address{365 Walker Dr.}{Mountain View, CA 94043}{USA}% optional, remove the line if not wanted; the "postcode city"and "country" arguments can be omitted or provided empty
\phone[fixed]{+1~(832)~628-4671}
\email{nneveu@slac.stanford.edu}
\extrainfo{nneveu@stanford.edu}
\makecvtitle

\section{Education}
\cventry{Chicago, IL}{Illinois Institute of Technology}{Ph.D Physics}{2013--2018}{College of Science}{}%{\textit{Grade}}{Description}  % arguments 3 to 6 can be left empty
\cventry{Houston, TX}{University of Houston}{B.S. Electrical Engineering}{2009--2013}{Cullen College of Engineering}{}

\iffalse
\section{Ph.D Thesis}
\cvitem{Title}{\emph{Design for  Staged Two Beam Acceleration at the Argonne Wakefield Accelerator}}
\cvitem{Advisors}{Linda Spentzouris, John Power}
\cvitem{Summary}{
	I discuss beam line design, simulation, and optimization, of a
	beam line with the potential for dielectric two beam acceleration.
    Preliminary and prepratory experimental measurements
    are included.    }
\fi

\section{Research Experience}
\cventry{Menlo Park, CA}{SLAC National Accelerator Laboratory}{Associate Scientist - Diagnostics Department}{October 2021--Present}{}{
    Contributing to LCLS operations and LCLS-II comissioning through research, maintenance, and leadership activities.
    \begin{itemize}
        \item Taking LCLS-II commissioning shifts for:
        \begin{itemize}
         \item wire scanner set up, HOMBOM, and testing of installed diagnostics.
         \item emittance GUI testing and measurements.
         \item beam transport down stream.
        \end{itemize}
        \item Training to become a SME on wire scanner commissioing and installations.
         \item Lead for Higher Order Mode Beam Offset Monitor project (LCLS-II injector).
         \begin{itemize}
             \item ASP submission, CAM, and management of project.
         \end{itemize}
        \item Taking shifts as program deputy.
        \item Responsible for emittance GUI maintainence and measurement training.
        \item  Responsible for tracking and pushing LCLS injector OTR repairs forward.
        \item Contributed to SLAC technical notes.
         \item Performed injector simulation studies for LCLS and LCLS-II.
         \begin{itemize}
             \item Two papers were published with this material.
             \item I have ongoing collaborations with ANL through this work.
         \end{itemize}
         \item Submitted LDRD on beam shaping for medical accelerators (was not funded).
         \item Service roles:
         \begin{itemize}
             \item Reviewer PRAB and NIM.
             \item On the SAGE Executive Committee.
             \item Particpated in planning and instruction at SLACPAS.
             \item Chair of Women's Employee Resource Group.
             \item Vice-Chair of APS Division of Beams - Education, Outreach, and Diversity Committee.
         \end{itemize}
    \end{itemize}
}

\vspace{0.5em}
\cventry{Menlo Park, CA}{SLAC National Accelerator Laboratory}{Research Associate -- Experimental ARD}{December 2018-- September 2021}{}{
Beam dynamics work focused on LCLS and LCLS-II with an emphasis on simulation and programming work in Python.
\begin{itemize}
          \item Optimization of Gaussian and Sum Frequency Generated laser pulses in LCLS-II injector.
          \begin{itemize}
              \item Positive results of this simulation work led to the laser group moving forward with new optics installation.
          \end{itemize}
          \item Worked with group members to create and inform machine learning models of LCLS and LCLS-II.
          \begin{itemize}
              \item The simulations and models are being used for machine prediction and furhter ML reasearch at SLAC.
          \end{itemize}
         \item Submitted proposal to Laser Improvement task force for new temporal/logitudinal laser diagnostic.
         \begin{itemize}
             \item This project was accepted, parts ordered, and awaiting installation.
         \end{itemize}
		\item Updated emittance measurement GUI for LCLS-II and non-relativistic beams.
		\item Participated in LCLS-II early injector commissioning: emittance and beam size measurements, and QE maps.
        \begin{itemize}
            \item these results were included in the LCLS EIC publication.
        \end{itemize}
        \item Created repository for general data processing Python tools for LCLS,\newline \url{https://github.com/slaclab/lcls-tools}.
        %\item Optimization of beam dynamics for Gaussian temporal profiles in LCLS-II commissioning planning.
		%\item Incorporation of transverse laser profiles in simulation using Virtual Cathode Camera images.
        %\item Working with group members to inform surrogate models (machine learning models).
        %\item Creating simulation and measured data sets for machine learning studies.
        \item Improved PIC simulation models for LCLS and LCLS-II which improved ML models.
         \item Contributed simulations and experimental runs at the X-band test facility in TID.
\end{itemize}
}

\vspace{1em}
\cventry{Lemont, IL}{Argonne National Laboratory (ANL)}{Graduate Student, Argonne Wakefield Accelerator (AWA)}{2013--2018}{}{
Optimization and simulation of RF photoinjectors at AWA using
the parallel PIC code, OPAL-T, \newline and experimental validation of results.
\begin{itemize}%
	\item Thesis: Design for Staged Two-Beam Acceleration (TBA) at the Argonne Wakefield Accelerator
	%\item Simulations:
		%\item TBA beam line design at 40 nC for AWA.
		%\item Large ensemble HPC simulations.
		\item High charge HPC linac optimization (40 nC) using NLopt Python package.
		\item Optical Transition Radiation (OTR) simulations and experiments at 1 nC.
		\item Simulation of the Euclid Techlabs thermionic gun design.
		\item Wrote post-processing script in Python for analyzing transverse beam images.
        \item Lead on testing of the high charge fast rise time kicker manufactured as a part of my thesis work.
		%\item Built and installed OPAL-T on two Linux clusters at ANL: Bebop and Theta.
		%\item \url{https://github.com/nneveu}
		\item Beam size, charge, energy, emittance, and RF measurements.
		%\item RF power measurements for linac tanks and gun.
		%\item 40 nC emittance and bunch length measurements.
		%\item Intensity improvement of UV laser pulse train.
		\item UV multisplitter and relay design, assembly, characterization, and improvement.
		%\item Design of relay imaging transport system for drive beam line.
\end{itemize}
}


\section{Publications}
% Publications from a BibTeX file without multibib
%  for numerical labels: \renewcommand{\bibliographyitemlabel}{\@biblabel{\arabic{enumiv}}}% CONSIDER MERGING WITH PREAMBLE PART
%  to redefine the heading string ("Publications"): \renewcommand{\refname}{Articles}

\nociteart{*}
\bibliographystyleart{ieeetr}
\bibliographyart{./cvparts/papers}
%\vspace{0.5em}

\nociterefproc{*}
\bibliographystylerefproc{ieeetr}
\bibliographyrefproc{./cvparts/refproc}
%\vspace{0.5em}

\nociteinprog{*}
\bibliographystyleinprog{ieeetr}
\bibliographyinprog{./cvparts/inprog}
%\vspace{0.5em}


%\section{Proceedings}
\nociteproc{*}
\bibliographystyleproc{ieeetr}
\bibliographyproc{./cvparts/proceedings}
%\vspace{0.5em}
%
%\newpage

\section{Research Presentations}
\subsection{Select Conference or Workshop} %Workshops, Symposium
\cvitem{}{Nonlinearly shaped pulses at LCLS-II, \textbf{NAPAC 2022}}
\cvitem{}{Multiobjective optimization of the LCLS-II photoinjector, \textbf{NAPAC 2022}}
\cvitem{}{Nonlinear shaped laser pulses in the LCLS-II photoinjector, \textbf{CLEO 2021}}
\cvitem{}{Benchmarking the LCLS-II photoinjector, \textbf{NAPAC 2019}}
\cvitem{}{Surrogate modeling for charged particle accelerator beam dynamics, \textbf{OPAL Retreat 2019}}
%\cvitem{}{Comparison of model based and heuristic optimization algorithms applied to photoinjectors using libensemble, \textbf{ICAP 2018}}
\cvitem{}{Study of space charge dominated beams at the AWA RF photoinjector, \textbf{Space Charge 2017}}

\subsection{Select Seminars and Colloquia}
\cvitem{}{An introduction to photoinjectors, \textbf{University of Houston 2023}}
%\cvitem{}{LCLS-II Optimization, \textbf{SLAC ARD 2020}}
\cvitem{}{Photoinjectors: from wakefields to free electron lasers, and the intersection of math and accelerator physics, \textbf{University of Hawaii at Manoa 2020}}
\cvitem{}{Modeling the LCLS-II injector: from optimization to machine learning, \textbf{LBNL 2019}}
%\cvitem{}{Emittance GUI Studies, \textbf{SLAC AD 2019}}
\cvitem{}{Fully staged two-beam acceleration at the Argonne Wakefield Accelerator, \textbf{MIT 2018}}
%\cvitem{}{Fully staged two-beam acceleration at the Argonne Wakefield Accelerator, \textbf{SLAC 2018}}
%\cvitem{}{Fully staged two-beam acceleration at the Argonne Wakefield Accelerator, \textbf{LANL 2018}}
\cvitem{}{Simulations, optimization, and experimental measurements at the AWA, \textbf{RadiaSoft 2018}}
\cvitem{}{Photoinjector simulations using OPAL-T for the Argonne Wakefield Accelerator, \textbf{PSI 2017}}
%\cvitem{}{Research at the Argonne Wakefield Accelerator, \textbf{ANL 2016}}
%\cvitem{}{Two-beam acceleration at the Argonne Wakefield Accelerator, \textbf{IIT Chicago 2016}}

%\newpage
\section{Select Awards}
\cventry{}{}{SLAC Spot Awards}{2019-2023}{}{}
\vspace{-1.5em}
\cventry{}{}{Argonne Training Program on Extreme Scale Computing (ATPESC)}{ 2018}{}{}
\vspace{-1.5em}
%\cventry{}{}{AAC 2018 Student Travel Grant}{ 2018}{}{}
%\vspace{-1.5em}
\cventry{}{Two million core hours awarded on Theta, for work done in collaboration with PSI.}{ANL Director's Discretionary Allocation}{ 2018}{}{}%}
%\cventry{}{}{IPAC 2018 Student Travel Award}{ 2018}{}{}%}
%\vspace{-1.5em}
%\cventry{}{}{IPAC 2017 Student Travel Award}{ 2017}{}{}%}
%\vspace{-1.5em}
\cventry{}{}{Phelps Short Course Grant Award}{ 2016}{}{}%}
\vspace{-1.5em}
\cventry{}{}{DOE Science Graduate Student Research Award}{2015--2016}{}{}%}
\vspace{-1.5em}
%\cventry{}{}{USPAS Scholarships}{2014-2018}{}{}
%\vspace{-1.5em}
%\cvlistitem{Microwave Measurements and Beam Instrumentation, Fundamentals of Timing and Synchronization}
%\cvlistitem{Vacuum Science and Technology, Accelerator Physics Using Maple}


%\newpage
\section{Teaching}
\cvitemwithcomment{SLACPAS}{Lab Instructor, Fundamentals of Accelerator Physics and Technology}{\textbf{\emph{ 2022}}}
\cvitemwithcomment{USPAS}{Instructor, UHV Free Electron Lasers }{\textbf{\emph{ 2021}}}
\cvitemwithcomment{USPAS}{Lab Instructor, Fundamentals of Accelerator Physics and Technology}{\textbf{\emph{ 2019}}}
\cvitemwithcomment{USPAS}{TA \& Guest Lecturer, Fundamentals of Accelerator Physics and Technology}{\textbf{\emph{ 2018}}}
\cvitemwithcomment{Vandercook School of Music}{Adjunct, Physics of Sound and Light}{\textbf{\emph{ 2018}}}
\cvitemwithcomment{Institute of Modern Physics}{TA, Technical English}{\textbf{\emph{ 2017}}}
\cvitemwithcomment{Tsinghua University}{TA, Technical English}{\textbf{\emph{ 2015-2017}}}
\cvitemwithcomment{USPAS}{TA, Fundamentals of Accelerator Physics and Technology}{\textbf{\emph{ 2015}}}
\cvitemwithcomment{IIT Chicago}{TA, Electromagnetism and Optics \& Instrumentation Laboratory}{\textbf{\emph{2013-2015}}}
%\vspace{0.5em}


\section{Mentorship \& Volunteer Work}
Actively involved in mentorship of graduate students, summer students, and co-workers. I also volunteer in educational outreach initiatives at SLAC, VEX robotics competitions, and Adopt-a-Physicist, \url{https://www.adoptaphysicist.org/}.
%\cvitemwithcomment{Paris Franz}{Graduate student, Stanford University}{\textbf{\emph{2020-Present}}}
%\cvitemwithcomment{Lipi Gupta}{Graduate student, University of Chicago}{\textbf{\emph{2019-Present}}}
%\cvitemwithcomment{Mirian Juan}{Graduate student, CSU Long Beach}{\textbf{\emph{2019-Present}}}
%\cvitemwithcomment{Lucas Berens, Amy Sihn, McCay Rhodebeck}{SULI Summer students}{\textbf{\emph{2020}}}

\section{Memberships}
\cvlistdoubleitem{American Physical Society}{Society of Women Engineers}
\cvlistdoubleitem{Institute of Electrical and Electronics Engineers}{}

\iffalse
\section{Recent Volunteer Work}
\cvitemwithcomment{Aug. 2022}{SAGE-S Summer Camp for Girls, Projects Lead, SLAC}{}
\cvitemwithcomment{Otc. 2020-2022}{Adopt-a-Physicist}{}
%\cvitemwithcomment{Feb. 2021}{Cabrillo High School STEM Talk}{}
%\cvitemwithcomment{Aug. 2019}{SAGE-S Summer Camp for Girls, SLAC}{}
%\cvitemwithcomment{Jan. 2020}{Science Bowl, SLAC}{}
%\cvitemwithcomment{Otc. 2017, 2018, 2020}{Adopt-a-Physicist}{}
%\cvitemwithcomment{Oct. 2019}{Community Day, SLAC}{}
%\cvitemwithcomment{Jun. 2019}{CORE Summer Science Institute, SLAC}{}
%\cventry{}{Hosted by ANL, participated as a mentor}{Introduce a girl to engineering day}{April 2017, Feb 2018}{}{}
%\cventry{}{Visited Spry CLHS and Glenbard East High School.}{Hour of Code}{Dec 2017}{}{}
%\cventry{}{Monroe School}{Science Fair Judge}{Feb 2017}{}{}


\section{Programming Languages}
\cvitem{Proficient in}{ASTRA, OPAL-T, Python, MATLAB}
\cvitem{Prior experience with}{IMPACT, Genesis, C/C++}

\fi

%\iffalse
%\section{References}
%\vspace{1em}
%
%\begin{enumerate}
%	\itemsep1em
%	\item Linda Spentzouris, IIT
%	\begin{itemize}
%		\item[] Phone: 312.567.3577
%		\item[] Email: linda@agni.phys.iit.edu
%	\end{itemize}
%	\item Andreas Adelmann, PSI
%	\begin{itemize}
%		\item[] Phone: 41.56.310.4233
%		\item[] Email: andreas.adelmann@psi.ch
%	\end{itemize}
%	\item Jeffrey Larson, MCS-ANL
%	\begin{itemize}
%		\item[] Phone: 630.252.3221
%		\item[] Email: jmlarson@anl.gov
%	\end{itemize}
%	\item Chunguang Jing, Euclid Techlabs, LLC
%	\begin{itemize}
%		\item[] Phone: 630.313.6562
%		\item[] Email: jingchunguang@gmail.com
%	\end{itemize}
%
%\end{enumerate}
%\fi
%
%\clearpage




\end{document}
