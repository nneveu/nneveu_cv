 As a graduate student, I worked in small facility, as a member of the Argonne Wakefield Accelerator (AWA) group
 at Argonne National Laboratory. My research focused on a beam line design for Two-Beam Acceleration experiments, and culminated
 in the fabrication and experimental testing of a high charge kicker. 
 I took advantage of the  high performance computing (HPC) resources at the lab, by learning from onsite computer scientist
 and mathematicians at the Argonne Leadership Computing Facility. 
 They encouraged me to explore optimization and HPC resources for my 
 beam line design studies. However, being in a small experimental group, 
 I also learned about accelerators in a hands-on fashion. From aligning UV optics
 to taking beam based measurements, I was involved in each step of the experimental process at the AWA.


Since transitioning to SLAC, 
I have continued doing research in the area that
 would become my favorite topic during graduate school; photoinjectors.
 I have enthusiastically contributed to several projects related to this topic at SLAC:
 Surrogate Modeling (SM) efforts, Early Injector Commissioning (EIC), and beam shaping. 
 The SM project aims to create machine learning models to accurately predict 
 beam dynamics parameters in real time. 
 I use the simulation experience I gained in graduate school and a variety of codes to generate data sets used in training, 
 and to inform model goals for the LCLS/LCLS-II injectors.
 
 As a part of EIC, I have used my beam dynamics foundation to update the 
 LCLS emittance Matlab Graphical User Interface to account for the low energy beams emitted by the 
 APEX-like gun that will feed the superconducting linac. I have also  
 taken beam shifts to acquire emittance and beam size data, that will be used 
 for benchmarking the EIC gun, and informing future SM work.
 For the beam shaping project, I have been investigating the effects of Gaussian vs. 
 Flattop laser profile distributions on photoinjector performance. 
 I will incorporate several realistic laser profiles in the temporal and transverse dimensions. 
 
 This work in combination of with my experience in graduate school has encouraged me 
 to apply for this position. It provides the unique opportunity to expand and investigate ideas 
 I would like to implement to help improve FEL performance. 
 These include better beam dynamics characterization with the use of multistart optimization methods and uncertainty quantification (UQ). 
I propose testing efficient optimization algorithms because they can significantly reduce the amount of time needed for design and online optimization, 
 UQ can be used to quantitatively inform  which accelerator parameters have the most impact on FEL performance. 
 Preliminary studies show these methods can dramatically reduce the time needed for beam dynamics optimization and tuning. 
 Details on my proposed work can be found in my research statement.
 
 Please feel free to contact me with any questions. Thank you for considering my application. 
 \vspace{0.5em}
 