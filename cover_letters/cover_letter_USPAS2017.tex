\iftrue
\recipient{USPAS}{Fermilab, MS 125\\Kirk Rd \& Wilson St.\\Batavia, IL  60510}
\date{August 15, 2017}
\opening{Dear Sir or Madam,}
\closing{Thank you,}
\enclosure[Attached]{curriculum vit\ae{}}          % use an optional argument to use a string other than "Enclosure", or redefine \enclname
\makelettertitle

I can not exaggerate the positive impact that USPAS has had on my graduate eduction and training, 
as each class provided crucial information that expanded my knowledge of accelerator physics. 
USPAS is unique in that it offers education in 
topics that can not be found at any university, and the lab intensive classes offer hands on training 
that is hard to come by. In addition to spectacular classes, every breakfast, lunch, homework session, 
and break is an opportunity to exchange ideas with scientist and engineers from labs across the country and the world.
These interactions, in combination with the course work, help me learn about what is going on in the field.
It can also be a time to learn some interesting history!

It is exciting to bring those ideas, new and old, back to my group at the 
Argonne Wakefield Accelerator (AWA) facility. As a PhD student, I started full time at the AWA 
after two years of classes. I then spent the next two years doing photoinjector 
simulations and experiments for the two electron beam lines at the AWA. 
I've also done UV optics work in order to improve the charge distribution in our bunch trains. 
The impact of USPAS classes in my work is tangible. The information and resources (the books are great) 
I acquired at USPAS have helped me find insight in my simulations, understand our timing and rf systems better, 
and appreciate the difficulty of maintaining a pristine vacuum as most facilities do. This information has helped me 
as I design a beam line for upcoming two beam acceleration experiments, as I work on our UV laser, and while
talking to vendors. USPAS has given me training in content and skills that are critical for anyone working
at an accelerator facility, and I hope to continue learning from this invaluable program.

\makeletterclosing
\fi 
