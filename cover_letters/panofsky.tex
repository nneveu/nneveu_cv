%Panofsky cover letter
Since transitioning to SLAC after graduate school last December, 
I have been captivated by the breadth of exciting projects here, 
and the talented people pushing them forward.
I have continued doing research in the area that
 would become my favorite topic during graduate school, photoinjectors, 
 by joining and contributing to several projects here. 
 These includes the Surrogate Modeling (SM) efforts, Early Injector Commissioning (EIC), and beam shaping. 
 The SM project aims to create machine learning models to accurately predict 
 beam dynamics parameters in real time. 
 I use the simulation experience I gained in graduate school and variety to codes to generate data sets used in training, 
 and to inform model goals for the LCLS/LCLS-II injectors.
 
 As a part of EIC, I have used my beam dynamics foundation to update the 
 LCLS emittance Matlab Graphical User Interface to account for the low energy beams emitted by the 
 APEX-like gun that will feed the superconducting linac. I have also recently 
 taken beam shifts to acquire emittance and beam size data, that will be used 
 for benchmarking the EIC gun, and informing work in the SM project mentioned above. 
 Being in the control room and a part of EIC has been a pleasure and exciting, 
 considering these are the first steps and the beginning of a large project.
 For the beam shaping project, I have been investigating the effects of Gaussian vs. 
 Flattop laser profile distributions on photoinjector performance through the first cryomodule. 
 I will incorporate several realistic laser profiles in the temporal and transverse dimensions. 
 
 This work in combination of with my experience in graduate school has encouraged me 
 to apply for the Panofsky Fellowship. It provides the unique opportunity to expand and investigate ideas 
 I would like to implement to help improve FEL performance at LCLS. 
 These include better beam dynamics characterization with the use of multistart optimization methods and uncertainty quantification (UQ). 
I propose testing efficient optimization algorithms because they can significantly reduce the amount of time needed for design optimization, 
 allowing for in depth and non-traditional beam dynamics studies of large systems.
 UQ can be used to quantitatively inform  which accelerator parameters have the most impact on FEL performance. 
 Preliminary studies show these methods can dramatically reduce the time needed for beam dynamics optimization and tuning. 
 This would help provide better light to users. Details on my proposed work can be found in my research statement.
 
 Please feel free to contact me with any questions regarding my application materials. Thank you for your time and for considering my application. 
 \vspace{1em}
 